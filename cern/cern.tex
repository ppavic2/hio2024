%%%%%%%%%%%%%%%%%%%%%%%%%%%%%%%%%%%%%%%%%%%%%%%%%%%%%%%%%%%%%%%%%%%%%%
% Problem statement
\begin{statement}[
  problempoints=100,
  timelimit=1.5 sekunda,
  memorylimit=512 MiB,
]{CERN}

CERN je međunarodna institucija fokusirana na nuklearna istraživanja i fiziku elementarnih čestica. 
Sustav akceleratora čestica u CERN-u koristi se za provođenje eksperimenata koji uključuju sudaranje čestica pri velikim brzinama. 

Promatramo $N$ čestica poredanih u niz. Svaka čestica određena je svojom \textit{vrstom} $v_i$, što 
predstavljamo prirodnim brojem između 1 i $N$. 

U najnovijem istraživanju potrebno je provesti $Q$ eksperimenata. U $i$-tom eksperimentu 
promatramo sve čestice od $l_i$-te do $r_i$-te u nizu $(l_i < r_i)$. Među promatranim česticama 
možemo odabrati bilo koje dvije čestice različite vrste te ih sudariti u akceleratoru, čime obje 
čestice bivaju uništene. Navedeni postupak sudaranja ponavljamo dok god među promatranim česticama 
postoje dvije čestice različite vrste. Eksperiment završava ili time što su sve promatrane čestice 
uništene, ili je preostao neki broj čestica iste vrste. Naravno, ovisno o tome kojim redoslijedom i 
koje čestice sudaramo, moguće je na kraju završiti s raznim vrstama čestica. 

Budući da sudaranje čestica nije jeftino, 
odlučili ste da ćete eksperimente provoditi samo u teoriji. Sada vas za svaki eksperiment zanima 
koliko postoji vrsta čestica tako da je moguće eksperiment završiti s nekim brojem preostalih 
čestica te vrste. 

%%%%%%%%%%%%%%%%%%%%%%%%%%%%%%%%%%%%%%%%%%%%%%%%%%%%%%%%%%%%%%%%%%%%%%
% Input
\subsection*{Ulazni podaci}

U prvom su retku prirodni brojevi $N$ i $Q$, redom broj čestica i broj eksperimenata.

U sljedećem je retku niz od $N$ brojeva $v_1, \dots, v_N$, redom vrste čestica. 

U $i$-tom od sljedećih $Q$ redaka je par od dva prirodna broja $l_i$ i $r_i$ 
($1 \leq l_i < r_i \leq N$) koji predstavljaju promatrani interval čestica u $i$-tom eksperimentu.  

%%%%%%%%%%%%%%%%%%%%%%%%%%%%%%%%%%%%%%%%%%%%%%%%%%%%%%%%%%%%%%%%%%%%%%
% Output
\subsection*{Izlazni podaci}

Za svaki od $Q$ eksperimenata u zasebni redak ispišite traženi 
broj vrsta čestica s kojima je moguće završiti eksperiment.

%%%%%%%%%%%%%%%%%%%%%%%%%%%%%%%%%%%%%%%%%%%%%%%%%%%%%%%%%%%%%%%%%%%%%%
% Scoring
\subsection*{Bodovanje}

U svim podzadacima vrijedi $2 \leq N \leq 500~000$ i $1 \leq Q \leq 500~000$.

{\renewcommand{\arraystretch}{1.4}
  \setlength{\tabcolsep}{6pt}
  \begin{tabular}{ccl}
   Podzadatak & Broj bodova & Ograničenja \\ \midrule
    1 & 13 & Vrijedi $v_i \leq 10$ za svaki $i = 1, \dots, N$. \\
    2 & 19 & Postoje najviše dvije čestice svake vrste. \\
    3 & 17 & $N, Q \leq 2000$ \\
    4 & 19 & $N, Q \leq 100~000$ \\
    5 & 32 & Nema dodatnih ograničenja. \\
\end{tabular}}

%%%%%%%%%%%%%%%%%%%%%%%%%%%%%%%%%%%%%%%%%%%%%%%%%%%%%%%%%%%%%%%%%%%%%%
% Examples
\subsection*{Probni primjeri}
\begin{tabularx}{\textwidth}{X}
\sampleinputs{test/cern.dummy.in.1}{test/cern.dummy.out.1}
\end{tabularx}

\textbf{Pojašnjenje probnog primjera:}

U prvom eksperimentu možemo sudariti čestice vrsta 3 i 4, čime preostaju dvije čestice vrste 2. 
Ne postoji način da na kraju preostane neka druga vrsta čestica. 

U drugom eksperimentu moguće je za svaku vrstu čestica postići da na kraju preostani neki broj 
čestica te vrste. 

U četvrtom i petom eksperimentu će neovisno o odabiru sudara na kraju preostati neki broj čestica 
vrste 4. 

%%%%%%%%%%%%%%%%%%%%%%%%%%%%%%%%%%%%%%%%%%%%%%%%%%%%%%%%%%%%%%%%%%%%%%
% We're done
\end{statement}

%%% Local Variables:
%%% mode: latex
%%% mode: flyspell
%%% ispell-local-dictionary: "croatian"
%%% TeX-master: "../hio.tex"
%%% End: